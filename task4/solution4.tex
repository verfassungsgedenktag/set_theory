\documentclass[12pt, a4paper]{article}

% кодировка и языки
\usepackage[T2A]{fontenc}
\usepackage[utf8]{inputenc}
\usepackage[english, russian]{babel}

% ams
\usepackage{amsthm}
\usepackage{amsfonts}
\usepackage{amsmath}
\usepackage{amssymb}

% окружения
\renewcommand{\qedsymbol}{\(\blacksquare\)}

\newtheoremstyle{task}%
    {}{}%
    {\slshape}{}%
    {\bfseries}{.}%
    { }{\thmname{#1}\thmnumber{ #2}\thmnote{ #3}}

\theoremstyle{task}
\newtheorem*{problem}{Задача}

% остальное
\frenchspacing
\binoppenalty=100000
\relpenalty=100000
\oddsidemargin=-0.5in
\textwidth=7.3in
\topmargin=-0.5in
\textheight=9.7in

% операторы
\DeclareMathOperator{\cov}{cov}
\DeclareMathOperator{\expected}{\mathbb{E}}
\DeclareMathOperator{\prob}{\mathbb{P}}
\DeclareMathOperator{\predicate}{P}
\DeclareMathOperator{\power}{\mathcal{P}}

\usepackage[bb=boondox]{mathalfa}
\usepackage{eucal}
\usepackage{physics}
\usepackage{enumitem}
\usepackage{centernot}
\usepackage{algpseudocodex}

\newcommand{\cA}{\mathcal{A}}
\newcommand{\cB}{\mathcal{B}}
\newcommand{\cC}{\mathcal{C}}

\newcommand{\bA}{\mathbb{A}}
\newcommand{\bC}{\mathbb{C}}
\newcommand{\bG}{\mathbb{G}}
\newcommand{\bN}{\mathbb{N}}
\newcommand{\bR}{\mathbb{R}}
\newcommand{\bQ}{\mathbb{Q}}
\newcommand{\bZ}{\mathbb{Z}}

\renewcommand{\emptyset}{\varnothing}


\begin{document}
    \section*{Домашнее задание №4, Марченко М.}

    \begin{problem}[1]
        Докажите, что следующие множества счётны: \(\bZ\), \(\bN \times \bN\), \(\bQ\), множество всех конечных последовательностей натуральных чисел, множество всех конечных подмножеств натурального ряда, \(\bQ [x]\), множество всех алгебраических чисел.
    \end{problem}
    \begin{proof}[Решение]
        Построим биекцию \(\varphi \colon \bN \to \bZ\): пусть \(0 \mapsto 0\), \(1 \mapsto 1\), \(2 \mapsto -1\), \(3 \mapsto 2\), \(4 \mapsto -2\), \ldots В общем случае \[
            \varphi(n) =
            \begin{cases}
                -\frac{n}{2}, & n \ \text{чётно}, \\
                \frac{n+1}{2}, & n \ \text{нечётно}.
            \end{cases}
        \]
        \qed

        Множество \(\bN \times \bN\) состоит из пар вида \((n, m)\). Пронумеруем их следующим образом: сначала такие, для которых \(n + m = 0\), потом такие, для которых \(n + m = 1\), и так далее для каждого натурального числа. \qed
        
        \(\bQ\) можно представить как множество пар вида \((p, q)\), где \(p \in \bZ, q \in \bN \setminus \{0\}\), в котором пары \((p, q)\) и \((r, s)\) считаются одинаковыми, если \(ps = qr\). Тогда сначала можно биективно отобразить \(\bZ\) в \(\bN\), а потом воспользовать построениями из предыдущего абзаца, пропуская при подсчёте копии. \qed

        Воспользуемся теоремой Кантора--Бернштейна: вложим сначала натуральные числа в множество всех конечных последовательностей натуральных чисел, а потом наоборот. Первое делается очевидным образом: сопоставим числу последовательность, состоящую из него самого. Обратное вложение можно построить так: пусть имеем последовательность \((a_n)\), запишем число разрядов в десятичной записи \(a_1\), а затем само число \(a_1\), после повторим для остальных членов последовательности. Получим некоторое натуральное число, по которому однозначно можно восстановить последовательность. Например, последовательности \(4, 0, 40, 4000\) соответствует число~\(\textbf{1} 4 \textbf{1} 0 \textbf{2} 40 \textbf{4} 4000\). \qed

        Каждое подмножество натурального ряда задаёт последовательность \((a_n)\) единиц и нулей, где \(a_n = 1\), если число \(n\) принадлежит рассматриваемому подмножеству, и \(a_n = 0\) в противном случае. Если мы рассматриваем только конечные подмножества, то они задают последовательности нулей и единиц, тождественно нулевые с некоторого момента. Читая эти последовательности с конца, получим числа в двоичной системе счисления, причём уникальные для каждого подмножества. Также ясно, что каждому натуральному числу можно сопоставить подмножество,~---~достаточно просто перевести его в двоичную систему счисления. \qed

        Каждому многочлену с рациональными коэффициентами по построенной выше биекции между \(\bQ\) и \(\bN\) можно сопоставить многочлен с натуральными коэффициентами, который, в свою очередь, сопоставляется с некоторой конечной последовательностью натуральных чисел. Например, многочлену \(4 + 10x + 12x^4\) соответствует последовательность \(4, 10, 0, 0, 12\). \qed

        Опять воспользуемся теоремой Кантора--Берштейна. Каждому алгебраическому числу сопоставим некоторый многочлен с рациональными коэффициентами, корнем которого оно является. Так как для любого алгебраического числа существует бесконечное количество многочленов, корнем которых оно является, отображение можно построить таким образом, что разные числа будут отображаться в разные многочлены. Обратное вложение получим, воспользовавшись биекцией между \(\bQ[x]\) и \(\bN\), так как каждое натуральное число является алгебраическим.
    \end{proof}

    \begin{problem}[2]
        Докажите, что:
        \begin{enumerate}[label=(\alph{*})]
            \item следующие множества континуальны: \(\power(\bN)\), \(2^{\bN}\), \(3^{\bN}\), \(\bN^{\bN}\), \(\bR\), \(\bC\), \(2^{\bN} \times 2^{\bN}\), \((2^{\bN})^{\bN}\), любой открытый интервал, множество всех точек плоскости, множество всех последовательностей вещественных чисел, множество всех шаров в трёхмерном пространстве, множество всех прямых на плоскости, множество всех непрерывных функций на \(\bR\).
            \item существует вещественное трансцендентное число. Каких чисел больше, алгебраических или трансцендентных и почему?
        \end{enumerate}
    \end{problem}
    \begin{proof}[Решение]
        \begin{enumerate}[label=(\alph{*})]
            \item Подмножества натурального ряда задают последовательность \((\epsilon_n)\) из единиц и нулей. Переведём её в двоичную дробь следующим образом: \[
                \begin{cases}
                    0,1 \epsilon_1 \epsilon_2 \ldots, & \epsilon_k = 1 \ \text{для всех} \ k \ \text{с некоторого места}, \\
                    0,0 \epsilon_1 \epsilon_2 \ldots, & \text{иначе}.
                \end{cases}
            \]
            Обратное вложение получим, взяв, если есть альтернатива, для каждой двоичной дроби представление, оканчивающееся нулями, и <<прочитав>> дробную часть как последовательность из единиц и нулей, задающую некоторое подмножество натурального ряда. \qed

            Между \(\power(\bN)\) как множеством подмножеств натурального ряда и \(2^{\bN}\) как множеством функций вида \(f \colon \bN \to \{0, 1\}\) существует очевидная биекция: каждому подмножеству \(X\) сопоставляется функция \[
                f(n) =
                \begin{cases}
                    1, & n \in X, \\
                    0, & n \not \in X,
                \end{cases}
            \]
            и по каждой такой функции однозначно восстанавливается подмножество. \qed
            
            TODO

        \end{enumerate}
    \end{proof}

    \begin{problem}[3]
        Докажите, что:
        \begin{enumerate}[label=(\alph{*})]
            \item если \(X \neq \emptyset\), то существование инъекции из \(X\) в \(Y\) равносильно существованию сюръекции из \(Y\) на \(X\).
            \item объединение любой последовательности счётных множеств счётно.
        \end{enumerate}
    \end{problem}
    \begin{proof}[Решение]
        \begin{enumerate}[label=(\alph{*})]
            \item Докажем необходимость. Пусть имеем инъекцию \(f \colon X \to Y\). Построим сюръекцию \(g \colon Y \to X\) следующим образом: если существует \(x\) такой, что \(y = f(x)\), положим \(g(y) = x\) (таким образом мы построили прообразы для любого \(x \in X\), так как \(f\) --- инъекция); если для некоторого \(y\) не существует подходящего \(x\), отобразим его в произвольный.
            
            Достаточность. Пусть имеем сюръекцию \(g \colon Y \to X\). Построим инъекцию \(f \colon X \to Y\), выбрав для каждого \(x\) произвольный \(y\) такой, что  \(g(y) = x\). Так как каждый \(y\) отображается только в один \(x\), построенное отображение и впрямь инъекция. \qed
            \item Достаточно <<построить таблицу>>, в каждой строке которой будет выписано соответствующее множество последовательности. Каждый элемент таблицы будет соответствовать элементу~\(\bN \times \bN\) (пусть повторяющиеся элементы таблицы соответствуют одному и тому же элементу~\(\bN \times \bN\)).
        \end{enumerate}
    \end{proof}

    \begin{problem}[4]
        Докажите, что следующие множества конечны или счётны: любое множество попарно не пересекающихся интервалов числовой прямой; любое множество попарно не пересекающихся открытых шаров в пространстве; любое множество попарно не пересекающихся букв T на плоскости; множество всех точек разрыва любой монотонной функции на \(\bR\).
    \end{problem}
    \begin{proof}[Решение]
        Пусть имеется произвольное множество попарно не пересекающихся интервалов числовой прямой. Каждый интервал задаётся парой центр--радиус. Для каждого интервала можно выбрать центр и радиус так, чтобы они были рациональными, а интервалы остались попарно не пересекающимися. Множество же, состоящее из пар рациональных чисел, конечно или счётно. \qed

        Аналогично каждый шар задаётся парой центр--радиус, которые можно выбрать рациональными. В данном случае имеем всего четыре параметра, так как центр определяется тремя координатами, но на кардинальность это не влияет. \qed

        TODO  \qed

        Монотонная функция на \(\bR\) имеет только разрывы первого рода, то есть скачки. Пусть \(x\) --- точка разрыва, и \(f(x) = y_1\), \(f(x-) = y_2\) (предел \(f(x-)\) существует, так как по условию функция монотонна), где \(y_1 \neq y_2\). Интервал \((y_1, y_2)\) содержит какое-нибудь рациональное число, которое мы и поставим в соответствие точке разрыва. А так как функция монотонна, каждой точке будет соответствовать уникальное рациональное число.
    \end{proof}

    \begin{problem}[5]
        Докажите, что: множество конечно в точности тогда, когда любая инъекция этого множества в себя является биекцией; мощность любого конечного множества меньше мощности любого бесконечного множества; любое бесконечное множество содержит счётное подмножество; не существует множества наибольшей мощности.
    \end{problem}
    \begin{proof}[Решение]
        Пусть множество \(X\) состоит из \(n\) элементов и пусть существует инъекция \(\varphi\) этого множества в себя, которая не является при этом биекцией, то есть существует \(x_0 \in X\) с пустым прообразом. Составим таблицу из \(n\) пар вида \((x, f(x))\). Так как в \(x_0\) ничего не отображается, он упоминается в таблице один раз, но тогда некоторый \(x_1 \in X\) упоминается в таблице как минимум три раза, что противоречит инъективности \(\varphi\).

        Достаточно показать, что если множество \(X\) бесконечно, то существует его инъекция в себя, которая не является при этом биекцией. А это очевидно, так как можно рассмотреть любую инъекцию \(\psi \colon X \to X \setminus \{x_0\}\) для произвольного \(x_0\). \qed

        Любое конечное множество вкладывается в любое бесконечное, причём любое бесконечное не вкладывается в любое конечное. \qed

        Пусть \(X\) --- некоторое бесконечно множество. Возьмём произвольный его элемент и пометим числом \(1\), потом возьмём любой из ещё не помеченных и пометим числом \(2\). Так как на каждом шаге будут непомеченные элементы, из помеченных сформируется счётное множество. \qed

        Достаточно воспользоваться теоремой Кантора, согласно которой мощность булеана любого множества больше мощности этого же множества.
    \end{proof}
\end{document}